\documentclass[12pt,a4paper]{report}
\usepackage{ucs} 
\usepackage[utf8]{inputenc}
\usepackage[russian]{babel}
\usepackage[OT1]{fontenc}
\usepackage{amsmath}
\usepackage{amsfonts}
\usepackage{amssymb}
\usepackage{graphicx}
\usepackage{cmap}					% поиск в PDF
\usepackage{mathtext} 				% русские буквы в формулах
%\usepackage{tikz-uml}               % uml диаграммы

% TODOs
\usepackage[%
  colorinlistoftodos,
  shadow
]{todonotes}

% Генератор текста
\usepackage{blindtext}

\usepackage[section]{placeins}

%------------------------------------------------------------------------------

% Подсветка синтаксиса
\usepackage{color}
\usepackage{xcolor}
\usepackage{listings}
 
 % Цвета для кода
\definecolor{string}{HTML}{B40000} % цвет строк в коде
\definecolor{comment}{HTML}{008000} % цвет комментариев в коде
\definecolor{keyword}{HTML}{1A00FF} % цвет ключевых слов в коде
\definecolor{morecomment}{HTML}{8000FF} % цвет include и других элементов в коде
\definecolor{captiontext}{HTML}{FFFFFF} % цвет текста заголовка в коде
\definecolor{captionbk}{HTML}{999999} % цвет фона заголовка в коде
\definecolor{bk}{HTML}{FFFFFF} % цвет фона в коде
\definecolor{frame}{HTML}{999999} % цвет рамки в коде
\definecolor{brackets}{HTML}{B40000} % цвет скобок в коде
 
 % Настройки отображения кода
\lstset{
language=C, % Язык кода по умолчанию
morekeywords={*,...}, % если хотите добавить ключевые слова, то добавляйте
 % Цвета
keywordstyle=\color{keyword}\ttfamily\bfseries,
stringstyle=\color{string}\ttfamily,
commentstyle=\color{comment}\ttfamily\itshape,
morecomment=[l][\color{morecomment}]{\#}, 
 % Настройки отображения     
breaklines=true, % Перенос длинных строк
basicstyle=\ttfamily\footnotesize, % Шрифт для отображения кода
backgroundcolor=\color{bk}, % Цвет фона кода
%frame=lrb,xleftmargin=\fboxsep,xrightmargin=-\fboxsep, % Рамка, подогнанная к заголовку
frame=tblr
rulecolor=\color{frame}, % Цвет рамки
tabsize=3, % Размер табуляции в пробелах
showstringspaces=false,
 % Настройка отображения номеров строк. Если не нужно, то удалите весь блок
numbers=left, % Слева отображаются номера строк
stepnumber=1, % Каждую строку нумеровать
numbersep=5pt, % Отступ от кода 
numberstyle=\small\color{black}, % Стиль написания номеров строк
 % Для отображения русского языка
extendedchars=true,
literate={Ö}{{\"O}}1
  {Ä}{{\"A}}1
  {Ü}{{\"U}}1
  {ß}{{\ss}}1
  {ü}{{\"u}}1
  {ä}{{\"a}}1
  {ö}{{\"o}}1
  {~}{{\textasciitilde}}1
  {а}{{\selectfont\char224}}1
  {б}{{\selectfont\char225}}1
  {в}{{\selectfont\char226}}1
  {г}{{\selectfont\char227}}1
  {д}{{\selectfont\char228}}1
  {е}{{\selectfont\char229}}1
  {ё}{{\"e}}1
  {ж}{{\selectfont\char230}}1
  {з}{{\selectfont\char231}}1
  {и}{{\selectfont\char232}}1
  {й}{{\selectfont\char233}}1
  {к}{{\selectfont\char234}}1
  {л}{{\selectfont\char235}}1
  {м}{{\selectfont\char236}}1
  {н}{{\selectfont\char237}}1
  {о}{{\selectfont\char238}}1
  {п}{{\selectfont\char239}}1
  {р}{{\selectfont\char240}}1
  {с}{{\selectfont\char241}}1
  {т}{{\selectfont\char242}}1
  {у}{{\selectfont\char243}}1
  {ф}{{\selectfont\char244}}1
  {х}{{\selectfont\char245}}1
  {ц}{{\selectfont\char246}}1
  {ч}{{\selectfont\char247}}1
  {ш}{{\selectfont\char248}}1
  {щ}{{\selectfont\char249}}1
  {ъ}{{\selectfont\char250}}1
  {ы}{{\selectfont\char251}}1
  {ь}{{\selectfont\char252}}1
  {э}{{\selectfont\char253}}1
  {ю}{{\selectfont\char254}}1
  {я}{{\selectfont\char255}}1
  {А}{{\selectfont\char192}}1
  {Б}{{\selectfont\char193}}1
  {В}{{\selectfont\char194}}1
  {Г}{{\selectfont\char195}}1
  {Д}{{\selectfont\char196}}1
  {Е}{{\selectfont\char197}}1
  {Ё}{{\"E}}1
  {Ж}{{\selectfont\char198}}1
  {З}{{\selectfont\char199}}1
  {И}{{\selectfont\char200}}1
  {Й}{{\selectfont\char201}}1
  {К}{{\selectfont\char202}}1
  {Л}{{\selectfont\char203}}1
  {М}{{\selectfont\char204}}1
  {Н}{{\selectfont\char205}}1
  {О}{{\selectfont\char206}}1
  {П}{{\selectfont\char207}}1
  {Р}{{\selectfont\char208}}1
  {С}{{\selectfont\char209}}1
  {Т}{{\selectfont\char210}}1
  {У}{{\selectfont\char211}}1
  {Ф}{{\selectfont\char212}}1
  {Х}{{\selectfont\char213}}1
  {Ц}{{\selectfont\char214}}1
  {Ч}{{\selectfont\char215}}1
  {Ш}{{\selectfont\char216}}1
  {Щ}{{\selectfont\char217}}1
  {Ъ}{{\selectfont\char218}}1
  {Ы}{{\selectfont\char219}}1
  {Ь}{{\selectfont\char220}}1
  {Э}{{\selectfont\char221}}1
  {Ю}{{\selectfont\char222}}1
  {Я}{{\selectfont\char223}}1
  {і}{{\selectfont\char105}}1
  {ї}{{\selectfont\char168}}1
  {є}{{\selectfont\char185}}1
  {ґ}{{\selectfont\char160}}1
  {І}{{\selectfont\char73}}1
  {Ї}{{\selectfont\char136}}1
  {Є}{{\selectfont\char153}}1
  {Ґ}{{\selectfont\char128}}1
  {\{}{{{\color{brackets}\{}}}1 % Цвет скобок {
  {\}}{{{\color{brackets}\}}}}1 % Цвет скобок }
}
 
 % Для настройки заголовка кода
\usepackage{caption}
\DeclareCaptionFont{white}{\color{сaptiontext}}
\DeclareCaptionFormat{listing}{\parbox{\linewidth}{\colorbox{сaptionbk}{\parbox{\linewidth}{#1#2#3}}\vskip-4pt}}
\captionsetup[lstlisting]{format=listing,labelfont=white,textfont=white}
\renewcommand{\lstlistingname}{Код} % Переименование Listings в нужное именование структуры


%------------------------------------------------------------------------------

\author{К.~С.~Бойцов}
\title{Программирование}
\begin{document}
\maketitle
\chapter{Основные конструкции языка}
%############################################################
\section{Задание 1}
\subsection{Задание}

Пользователь задает длину отрезка в метрах. Вывести длину того же отрезка в саженях, аршинах и вершках.

\subsection{Теоретические сведения}

Воспользуемся следующим условием: \texttt{1 сажень = 3 аршина = 48 вершков, 1 вершок = 4.445 см.}

Для реализации данного алгоритма были использованы функции стандартной библиотеки, прототипы которых находятся в файле stdio.h, для ввода и вывода информации и math.h для выполнения необходимых вычислений.


\subsection{Проектирование}

В ходе проектирования было решено выделить 3 функции:

	\begin{itemize}
	

		\item void ui\_ meter\_ to\_ sazhen()
		
		 Пользователь вводит действительное \texttt{double} число метров, после чего вызывается конвертирующая функция.
		 
		 \item void meter\_ to\_ sazhen(double length)
		 
		 Функция, конвертирующая метры в сажени, аршины и вершки и выводящая результат в консоль.
		 Параметром функции является длина отрезка в метрах типа \texttt{double}. 
		 
		 \item void first\_ task\_ text()
	
	Функция выводит текст задания в консоль в меню пользовательского взаимодействия.
		 				
	\end{itemize}

\subsection{Описание тестового стенда и методики тестирования}
Среда разработки QtCreator 3.5.0, компилятор gcc (Ubuntu 5.2.1-22ubuntu2) 5.2.1 20151010, операционная система Ubuntu 15.10.

Для тестирования работы программы были выполнены статический анализ, также было проведено автоматической тестирование.

\subsection{Тестовый план и результаты тестирования}
		Для статического анализа использовалась утилита Cppcheck.
		
		\vspace{\baselineskip}
		 Cppcheck выдал незначительные предупреждения.
		
		\vspace{\baselineskip}
		При автоматическом тестировании вызывалась функция meter\_ to\_ sazhen, затем полученные значения сравнивались с ожидаемыми значениями. Результаты тестирования представлены в листингах.
		
		\vspace{\baselineskip}
	 

 
\subsection{Выводы}

При выполнении задания я отработал свои навыки в работе с основными конструкциями языка и получил опыт в организации функций одной программы.

\subsection*{Листинги}

meter\_ to\_ sazhen.c
\lstinputlisting[]
{../sources/subdirproject/lib/meter_to_sazhen.c}

\vspace{\baselineskip}

ui\_ meter\_ to\_ sazhen.c
\lstinputlisting[]
{../sources/subdirproject/app/ui_meter_to_sazhen.c}


\section{Задание 2}
\subsection{Задание}

Определить, пройдет ли кирпич со сторонами $а$, $b$, $c$ сквозь прямоугольное отверстие в стене со сторонами $r$ и $s$. Стороны отверстия должны быть параллельны граням кирпича. 

\subsection{Теоритические сведения}

В ходе выполения задания использовалась конструкция if...else. Кроме того, были применены функции стандартной библотеки из заголовочного файла stdio.h для ввода и вывода информации.

\subsection{Проектирование}


В ходе проектирования были выделены 5 функций:

\begin{itemize}

	\item void input\_ abc(int * a, int * b, int * c)

	Параметрами функции являются 3 целых значения.	
	Пользователю предлагается ввести 3 измерения кирпича.
	
	\item void input\_ hole(int * r, int * s)
	
	Параметрами функции являются 2 целых значения.
	Пользователю предлагается ввести 2 измерения отверстия.
	
	\item int brick(int length, int width, int height, int hole\_ length, int hole\_ width)
	
	Функция имеет 5 параметров: измерения кирпича и измерения отверстия.
	В функции происходят сравнения величин, после чего она возвращает либо удовлетворяющий набор условий, либо неудовлетворяющий.
	
	\item void ui\_ brick()
	
	Функция в зависимости от возвращенного значения функции \texttt{brick} выводит в консоль ответ на заданный вопрос.
	
	\item void second\_ task\_ text()
	
	Функция выводит текст задания в консоль в меню пользовательского взаимодействия.
	
	
	
\end{itemize}


\subsection{Описание тестового стенда и методики тестирования}
Среда разработки QtCreator 3.5.0, компилятор gcc (Ubuntu 5.2.1-22ubuntu2) 5.2.1 20151010, операционная система Ubuntu 15.10.

Для тестирования работы программы был выполнены статический анализ.

\subsection{Тестовый план и результаты тестирования}

		Для статического анализа использовалась утилита Cppcheck.
		
		\vspace{\baselineskip}
		
 Cppcheck выдал незначительные предупреждения.		
\subsection{Выводы}

При выполнении задания я получил опыт в организации функций одной программы.

\subsection*{Листинги}
brick.c
\lstinputlisting[]
{../sources/subdirproject/lib/brick.c}

\vspace{\baselineskip}

ui\_ brick.c
\lstinputlisting[]
{../sources/subdirproject/app/ui_brick.c}

%############################################################


\chapter{Циклы}
\section{Задание 1}
\subsection{Задание}

Поменять порядок цифр заданного натурального числа на обратный.Пример: 7283916 > 6193827.Строковые функции не использовать.

\subsection{Теоритические сведения}

В ходе выполения задания для произведения необходимых вычислений и преобразований использовались операции деление "\textbackslash" и деление с остатком "\%". Также использовался цикл while. Кроме того, были применены функции стандартной библотеки из заголовочного файла stdio.h для ввода и вывода информации.


\subsection{Проектирование}

В ходе проектирования были выделены 3 функции:

\begin{itemize}
	\item void ui\_ swapper()

	Функция считывает из консоли целое число, которое нужно развернуть, вызывает разворачивающую функцию и выводит готовое число.
	
	\item int swapper(int number\_ to\_ swap)
	
	Параметром функции является целое число, которое нужно развернуть.
	В функции происходит разворот числа и возвращение его значения.
	
	\item void third\_ task\_ text()
	
	Функция выводит текст задания в консоль в меню пользовательского взаимодействия.
	
	
\end{itemize}


\subsection{Описание тестового стенда и методики тестирования}

Среда разработки QtCreator 3.5.0, компилятор gcc (Ubuntu 5.2.1-22ubuntu2) 5.2.1 20151010, операционная система Ubuntu 15.10.

Для тестирования работы программы были выполнены статический анализ, также было проведено автоматическое тестирование.

\subsection{Тестовый план и результаты тестирования}

	Для статического анализа использовалась утилита Cppcheck.
	
	\vspace{\baselineskip}
	 Cppcheck выдал незначительные предупреждения.
	
	

	\vspace{\baselineskip}
	
	В ходе автоматического тестирования  вызывалась функция swapper. Результаты тестирования предоставлены в листингах.
	 \vspace{\baselineskip}
 

\subsection{Выводы}
В ходе выполнения я отработал навыки работы с циклами.
\subsection*{Листинги}
swapper.c
\lstinputlisting[]
{../sources/subdirproject/lib/swapper.c}

\vspace{\baselineskip}

ui\_ swapper.c
\lstinputlisting[]
{../sources/subdirproject/app/ui_swapper.c}

%############################################################

\chapter{Массивы}
\section{Задание 1}
\subsection{Задание}

В матрице $X(m,n)$ каждый элемент (кроме граничных) заменить суммой непосредственно примыкающих к нему элементов по вертикали, горизонтали и диагоналям.

\subsection{Теоритические сведения}

Для выполнения задания использовался цикл for, конструкция if...else, а также функции стандартной библиотеки из заголовочного файла stdlib.h для динамического выделения и освобождения памяти и stdio.h для ввода, вывода информации и работы с файлами.

\subsection{Проектирование}

Ввод и вывод данных реализован с помощью файлов. Входной файл должен содержать матрицу из целых чисел. 
 
В ходе проектирования были выделены следующие функции:

\begin{itemize}
 	
 	\item void ui\_ matrix()
 	
 	Пользователь вводит нужное ему количество столбцов и рядов из исходной матрицы, после чего создаётся и выводится в консоль созданная матрица.
 	
 	\item void matrix(int coloumns, int rows)
 	
 	Параметрами функции являются целые количества столбцов и рядов.
 	В функции выделяется память, генерируются и выводятся исходная и новая матрицы, после чего память освобождается.
 	
 	\item void fourth\_ task\_ text()
	
	Функция выводит текст задания в консоль в меню пользовательского взаимодействия.
 	
 	
\end{itemize}
	
\subsection{Описание тестового стенда и методики тестирования}
Среда разработки QtCreator 3.5.0, компилятор gcc (Ubuntu 5.2.1-22ubuntu2) 5.2.1 20151010, операционная система Ubuntu 15.10.

Для тестирования работы программы был выполнены статический анализ.
\subsection{Тестовый план и результаты тестирования}

Для статического анализа использовалась утилита Cppcheck.

\vspace{\baselineskip}
Cppcheck выдал незначительные предупреждения.

\subsection{Выводы}

При выполнении задания я понял принцип организации программы при работе с выделением динамической памяти, научился работать с файлами.

\subsection*{Листинги}

matrix.c
\lstinputlisting[]
{../sources/subdirproject/lib/matrix.c}

\vspace{\baselineskip}

ui\_ matrix.c

\lstinputlisting[]
{../sources/subdirproject/app/ui_matrix.c}

%############################################################

\chapter{Строки}
\section{Задание 1}
\subsection{Задание}

Текст, не содержащий собственных имен и сокращений, набран полностью прописными русскими буквами.Заменить все прописные буквы, кроме букв, стоящих после точки, строчными буквами.


\subsection{Теоритические сведения}

Для выполнения задания использовался цикл for, конструкция if...else, а также функции стандартной библиотеки из заголовочного файла ctype.h для  работы со строками и stdio.h для ввода, вывода информации.

\subsection{Проектирование}

В ходе проектирования были выделены следующие функции:

\begin{itemize}
	 \item void ui\_ lower()
	 
 	Инициализируется массив из символов, после чего поочерёдно вызываются две следующие функции и выводится конечный ответ.
 	
 	\item void input\_ lower(char * str)
 	
 	Пользователю предлагается ввести строку, в которой все буквы будут прописными.
 	
 	\item void lower\_ case(char * str)
 	
 	Все прописные буквы в предложении, кроме первой,заменяются на строчные.
 	
 	\item void fifth\_ task\_ text()
	
	Функция выводит текст задания в консоль в меню пользовательского взаимодействия. 
\end{itemize}
	
	
\subsection{Описание тестового стенда и методики тестирования}
Среда разработки QtCreator 3.5.0, компилятор gcc (Ubuntu 5.2.1-22ubuntu2) 5.2.1 20151010, операционная система Ubuntu 15.10.

Для тестирования работы программы был выполнены статический анализ.
\subsection{Тестовый план и результаты тестирования}

Для статического анализа использовалась утилита Cppcheck.

\vspace{\baselineskip}
 Cppcheck не выдал ошибок.

\vspace{\baselineskip}

\subsection{Выводы}

При выполнении задания я научился пользоваться функциями для работы со строками.
\subsection*{Листинги}
lower.c
\lstinputlisting[]
{../sources/subdirproject/lib/lower.c}

\vspace{\baselineskip}

ui\_ lower.c
\lstinputlisting[]
{../sources/subdirproject/app/ui_lower.c}

%############################################################

\chapter{Инкапсуляция}
\section{Задание 1}
\subsection{Задание}

Реализовать класс РАЦИОНАЛЬНОЕ ЧИСЛО (представимое в виде m/n). Требуемые методы: конструктор, деструктор, копирование, сложение, вычитание, умножение, деление, преобразование к типу double.

\subsection{Теоритические сведения}

Для выполнения задания использовался цикл for, конструкция if...else, а также класс exception стнадартной бибилиотеки.

\subsection{Проектирование}


В ходе проектирования программы было решено создать класс, который называется RationalNum.
Созданный класс содержит 2 поля с модификатором доступа private:

\begin{itemize}
	\item  int numerator;
    \item  int denominator;
    Числитель и знаменатель рационального числа.
\end{itemize}
	
	В классе определен конструктор 
\begin{itemize}

	\item RarionalNum(int numerator = 1, int denominator = 8)
	
	Конструктор со значениями по умолчанию для числа. 
	
			
\end{itemize}	
В классе определены 5 методов с модификатором доступа public:		

\begin{enumerate}	
	\item void Copy(RationalNum);
	
	Метод, аналогичный конструктору копирования.
	
	\item void Sum(int);
	
	Метод обеспечивает сложение. 
	
	\item void Multi(int);
	
	Метод обеспечивает умножение.
	
	\item void Divide(int);
	
	Метод обеспечивает деление.
	
	\item double ToDouble();
	
	Метод преобразует рациональное число к типу double. Возвращает соответственно это число.
	
	Так же перегружены операторы сложения, умножения и деления.

\end{enumerate}
	Так же был создаы класса исключений:
	
		\begin{itemize}
		\item DevNull
		Исключение вызывается, когда совершается попытка деления на ноль.
		\end{itemize}
		
		
\subsection{Описание тестового стенда и методики тестирования}
Среда разработки QtCreator 3.5.0, компилятор gcc (Ubuntu 5.2.1-22ubuntu2) 5.2.1 20151010, операционная система Ubuntu 15.10.

Для тестирования работы программы были выполнены статический анализ, также было проведено автоматической тестирование.

\subsection{Тестовый план и результаты тестирования}

Для статического анализа использовалась утилита Cppcheck.

\vspace{\baselineskip}
Cppcheck ошибок не обнаружил.
\vspace{\baselineskip}

Результаты автоматического тестирования представлены в листингах.

\subsection{Выводы}

При выполнении задания я понял принцип инкапсуляции и организации полей и методов класса.

\subsection*{Листинги}
rat\_ num.h
\lstinputlisting[]
{../sources/subdirproject/cpp-lib/rat_num.h}

\vspace{\baselineskip}

rat\_ num.cpp
\lstinputlisting[]
{../sources/subdirproject/cpp-lib/rat_num.cpp}

\chapter {Приложение}

\section{Автоматические тесты}

	\lstinputlisting[]{../sources/subdirproject/cpptest/tst_cpptesttest.cpp}
	
	
	\lstinputlisting[]{../sources/subdirproject/testtest/tst_testtesttest.cpp}

\end{document}
